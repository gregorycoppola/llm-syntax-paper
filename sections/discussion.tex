\label{sec:discussion}

\subsection{Zero-Shot Parsing and Negative Results}
We have replicated and expanded upon certain {\em negative} results from the literature pertaining to whether or not LLM's can do {\em zero-shot} parsing straight to the CoNLL format, to to ``annotate'' in contexts where the prompt is incorrect.

We have seen that {\em prompt strategy matters}, and that an overall {\em type} of annotation (e.g., prepositional phrase attachment) that can be gotten annotated accurately through one {\em prompting style} might be totally different in a different prompting style.

\subsection{Promise of LLM's a Linguists}
* it works
* lots of evdidence
* currently a prototype
* can be expanded into a real product

\subsection{Implications for Semantic Parsing}
* should work
* need to combine 1) it's a linguist, plus 2) semantic outlook

\subsection{Implications for Information Retrieval}
* can use Logic
* can move beyond just the llm