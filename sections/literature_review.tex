% Literature review section content will go here 
Recent work has explored the ability of large language models (LLMs) to serve as syntactic annotators, generating linguistic labels such as part-of-speech (POS) tags, dependency trees, and constituency parses. These models can operate in zero-shot, few-shot, or fine-tuned settings, and have shown varying degrees of success depending on the syntactic task, language, and model scale.

\subsection{POS Tagging and Shallow Syntax}

\citet{blevins2023llmpos} examined GPT-3 and GPT-J models on POS tagging, NER, and chunking tasks. Few-shot prompting with GPT-J 6B reached 79% accuracy for POS tagging, with GPT-3 (175B) scoring slightly lower. While this lags behind traditional supervised taggers (\textasciitilde96%), it demonstrates the models' internalized syntactic knowledge.

\citet{lai2023chatgptpos} used zero-shot prompting with ChatGPT across 17 languages. Using prompts in either English or the target language, they achieved average POS accuracy of 84--85%, outperforming fine-tuned XLM-R in most languages.

\citet{machado2024portpos} evaluated GPT-3, LLaMA-7B, and a Brazilian Portuguese LLM on POS tagging. With 10-shot prompts, GPT-3 achieved 90% accuracy, significantly outperforming the smaller models.

\subsection{Constituency Parsing}

\citet{bai2023llmconst} tested GPT-3.5, GPT-4, and LLaMA models on English constituency parsing. GPT-4 achieved an F1 of 73 with five-shot prompting, while open-weight LLaMA models scored below 30 F1.

\citet{tian2024chunkprompt} introduced a chunk-then-parse prompting approach using chain-of-thought (CoT) reasoning. This improved GPT-4's parsing quality on English and Chinese datasets, recovering deeper tree structures.

\subsection{Dependency Parsing}

\citet{hromei2024udllama} fine-tuned LLaMA-2 models on UD data across 26 languages. Their 13B model achieved \textasciitilde93--94% UAS, rivaling traditional biaffine parsers. Even the 7B model performed competitively.

Conversely, \citet{dubey2025zeroshotdep} found that most LLMs failed in zero-shot dependency parsing. Only LLaMA-70B marginally outperformed trivial baselines such as left-branching.

\subsection{Summary of Methods and Trends}

\begin{table}[ht] \centering \begin{tabular}{p{3.3cm} p{2cm} p{2cm} p{2.5cm} p{3.5cm}} \toprule \textbf{Study} & \textbf{Model} & \textbf{Task} & \textbf{Method} & \textbf{Result / Remark} \ \midrule Blevins et al. (2023) & GPT-3, GPT-J & POS, Chunk & Few-shot & Up to 79% POS (GPT-J); below SOTA \ Lai et al. (2023) & ChatGPT & POS (UD) & Zero-shot & Avg. 85% accuracy (17 langs) \ Machado & Ruiz (2024) & GPT-3 & POS (PT-BR) & Few-shot & 90% POS accuracy \ Bai et al. (2023) & GPT-4, LLaMA & Constituency & Few-shot & GPT-4: 73 F1; LLaMA: \textless30 F1 \ Tian et al. (2024) & GPT-4, GPT-3.5 & Constituency & Chunk + CoT & Improved tree depth and quality \ Hromei et al. (2024) & LLaMA-2 (13B) & Dependency & Fine-tuned & 93--94% UAS (26 langs) \ Dubey et al. (2025) & GPT-4, LLaMA-70B & Dependency & Zero-shot & Often \textless baseline \ \bottomrule \end{tabular} \caption{LLMs for syntactic annotation: task, model, method, and performance summary.} \label{tab:llm_syntax} \end{table}

These studies demonstrate that LLMs can approximate human-level syntactic annotation, particularly for POS tagging. However, performance in parsing (especially full dependency trees) remains variable without fine-tuning. Prompt design, instruction tuning, and model scale all contribute to improved results, and multilingual evaluation remains a critical benchmark for future progress.


