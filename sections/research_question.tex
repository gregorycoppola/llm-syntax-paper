\label{sec:contributions}
\subsection{Popularity of LLM's}
This paper is motivated by both the potential and the limits of {\em large language models} (LLM's).
On the one hand, LLM's have transformed many aspects of the field of computer science and the industry.
Many new kinds of workflows are being automated using {\em agentic} strategies.
The thesis of all of this work is that LLM's can {\em automate} complex workflows.
But, there are limitations at the same time.

\subsection{Limits of LLM's}
Large language models have well-known problems with {\em reasoning}.
Reasoning is not exact with them.
Reasoning is a mess.
Chain-of-Thought is not a well-defined concept.

Extending this with Reinforcement Learning has helped but also had limitations.
For one thing, you need a ``verifiable'' reward function.
This only works inherently in one way to start.
They are working on finding another way called ``generative reward models.''
That is actually what I proposed.
In that case, we can try another approach: which is to just do the entire {\em generative} model as a discrete logical model.

This requires {\em translating} the representation space.
The basic {\em representation space} of a LLM is the {\em token}.
We propose that the basic {\em unit} of modeling {\em reasoning} should be the {\em thought}---though the question then becomes how to define the {\em thought}.

We further propose for this that the way to represent a {\em thought} is as a sentence written in a {\em formal logic}, which is to say either {\em first-order logic}, or some close derivative of that.
Just in general, we can definitely have more {\em reliable} logical reasoning and inference if this is done in a symbolic space, as this isn not just more {\em interpretable}---to allow debugging and testing of the reasoning process---but also allows us to {\em specificy} rules.

In some ways, these are arguably two {\em sides of the same coin}---humans can {\em interpret} that which is {\em discrete}---because only when things are {\em categorized} can we form rules about them.

\subsection{Leveraging LLM's for Semantic Parsing}